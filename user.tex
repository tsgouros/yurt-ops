\documentclass[11pt]{article}
\usepackage{mathptmx}
\usepackage{verbatim}
\usepackage[margin=1.25in]{geometry}
\usepackage{graphicx}
\usepackage{color}
\usepackage{url}
\usepackage[colorlinks,
            bookmarks=true,
            hyperindex,
            linkcolor={blue},
            pdftitle={YURT Operations Manual}]{hyperref}

\newsavebox{\notebox}
\newenvironment{note}[1][Note]{\begin{lrbox}{\notebox}%
    \begin{minipage}{0.9\columnwidth}\textcolor{red}{\textbf{#1}:~}}%
    {\end{minipage}\end{lrbox}\begin{center}\setlength{\fboxsep}{8pt}%
    \fbox{\usebox{\notebox}}\end{center}}


% \newenvironment{note}[1][Note]{%
%   \begin{center}\setlength{\fboxsep}{10pt}
%   \framebox\bgroup\begin{minipage}{0.9\columnwidth}\textbf{#1}:~}%
%   {\end{minipage}\egroup\end{center}}


%{\textbf{#1}}{}

\newcommand{\yurt}{YURT}

\begin{document}

\title{Brown University YURT Operations Manual}
\author{Tom Sgouros}
\maketitle

\tableofcontents

\section{Introduction: Who is this manual for?}

There are two sections to this manual.  The first is for users who
simply wish to run software demos in the \yurt.  These are programs
that are either packaged as kiosk buttons or as scripts.

The other class of users is people who want to develop software to run
in the \yurt.  These people should already have a CCV user account that
belongs to the `graphics' group.

\section{Running the \yurt}

\subsection{Running Demos with the Kiosk}

The easiest way to run anything in the \yurt is with the ``kiosk,''
the graphical user interface available on the touch-screen computer at
the \yurt entrance.


\section{Developing \yurt Software}

\subsection{\yurt Architecture}

\subsection{Available Frameworks}

\begin{description}

\item[vrg3d]

\item[vrui]

\item[Blender]

\item[Writing3D]

\end{description}


\end{document}
